\documentclass[letterpaper]{article}
\usepackage[utf8]{inputenc}
\usepackage{authblk}
\usepackage{hyperref}
\usepackage[usenames,dvipsnames,svgnames,table]{xcolor}

\definecolor{myColor}{HTML}{006699}

\hypersetup {
  colorlinks = true, linkcolor = myColor, citecolor = myColor, urlcolor = myColor,
  pdfauthor = {Desjardins-Proulx, Philippe},
}

\renewcommand\Affilfont{\itshape\small}
\setcounter{section}{-1} % Sections too start at 0 :P

\begin{document}

\title{Restrictive Licenses for Code and Data Hinder Science}
% Find better title
\author[0,1,2]{Philippe Desjardins-Proulx}
\affil[0]{email: \href{mailto:philippe.d.proulx@gmail.com}{philippe.d.proulx@gmail.com}}
\affil[1]{Quebec Center for Biodiversity Science, Canada.}
\affil[2]{Universit\'e du Qu\'ebec, Canada.}
\date{\today}
\maketitle

\section{Introduction}

% Importance of data / code. Data mining, more collaboration = more problems with licenses.

We define restrictive licenses as those that put many restrictions on code/data
reuse, and oppose them to liberal licenses with almost no restrictions. For
data, the CC0 and CC-BY are liberal, unlike the creative commons with the NC
(non-commercial) or SA (share-alike) clauses. For code, the MIT/BSD/Apache
licenses are generally considered permissive and liberal, while the GPL is a
copy-left license that puts many restrictions on code reuse (see BOX 1). The
temptation to add restrictions can be strong. After all, programmers spend
hours building high-quality software, and field scientists spend equally long
hours collecting data. Why would you allow your code to be used without 
restrictions?

The question of licensing invariably leads to complex philosophical questions.
Does \emph{open} means to let others do as they like with data and code? Or,
does \emph{open} means we need to restrict usage so the data/code would never
be used in a closed environment? Instead of rehashing this debate, we will
focus on two practical problems with restrictive licenses. (1) Restrictive
licenses cause problems downstream, something we cannot afford in a world of
Big Data and massive data-sets. (2) Restrictive licenses are mired with legal
ambiguities. They are complex to understand, and complex to apply across
countries with different legal systems.

\section{Downstream problems}

\section{The case for simplicity}

Most of us would rather do science and focus on solving problems than discuss
legal issues.  And licenses can be really complex, potentially different
countries having different interpretations.  Recently, a German count

Liberal licenses are simple. Code under the MIT or BSD licenses can be used by
anyone: liberal license, GPL users, and proprietary. Restrictive open licenses
are often less accessible than proprietary code/data: if you want to build a
program with your MIT code, you cannot use the GNU Scientific Library (REF)
without adopting the GPL for your executable, but you can use the proprietary
Intel Math Kernel.

Manuscripts can be written openly with git \cite{ram13}.

% I looked at a bunch of trending repositories on github and 100% of them were under a liberal
% license. I'll get more data, but it seems there is a pretty strong trend toward liberal
% licenses. 
% -- PhDP

\bibliography{../refs}
\bibliographystyle{plain}

\end{document}

